\documentclass{article}
%\bibliographystyle{splncs}
\usepackage[dvips]{graphicx}
\usepackage{url}
\usepackage{tablefootnote}
% comment
% comment 2
\evensidemargin 0.1cm 
\oddsidemargin 0.1cm 
\topmargin 0in 
\textwidth 16.5cm 
\headheight -0.5in 
\headsep 0in 
\textheight 10in 
%\footskip 0.3in
%\footheight 0.1in 

\renewcommand{\baselinestretch}{1.5} 

\begin{document}

\title{Multimodal Identification of Birds from Visual and Acoustic Data}
\author{Arnav Bhavsar, Dileep A.~D.~, Padmanabhan Rajan\footnote{Author names
appear in alphabetic order.}\  \\Multimedia Analytics and Systems Group \\ 
School of Computing and Electrical Engineering\\
Indian Institute of Technology Mandi, India\\
\texttt{\{arnav, addileep, padman\}@iitmandi.ac.in}}

%\date{}
\maketitle
%\begin{abstract}
%\end{abstract}

\section*{Revision}

This is a revised version of the proposal, incorporating the comments made by
the external reviewers and the evaluation committie. The budget is modified in
this version of the document.

\section{Introduction}

% this is a seed grant proposal

% this is my second comment to create more confusion.
% GIT is full of magic, and there is water dripping from our roof.

% This is added after Arnavs comment

% Hopefully the final test, this is.


%Biodiversity is central to existence of most natural ecosystems \cite{}, and clearly the impact of human settlements (notwithstanding the nobility of their cause) can create disturbances and large variations in the biodiversity. 
The realization about the importance of biodiversity preservation has provided
impetus to the scientific study of ecological systems, including those for
monitoring the diversity, populations, distributions etc.
\cite{monitor1,monitor2,monitor3,monitor4}. 
%Some of the primary factors in studying ecosystems is observation and identification \cite{}. 
The development of such scientific studies occurred well before the common
availability of advanced technological aids. Thus, traditionally, the
observation and identification process has been manual requiring much time and
effort \cite{monitor4,human1,human2}. 

With the advent of the digital acquisition devices for capturing sights and
sounds, the observation aspect has become easier, faster and more reliable.
However, the identification component is still largely manual, with experts such
as zoologists, ornithologists, botanists etc.~manually labeling the acquired data. 

On the other hand, automated identification, labeling, annotating etc. has
become commonplace in various human technological endeavors such as biometrics,
surveillance, entertainment, navigation, sports etc. Clearly, given the nature
of technical progress in more human-centric domains, it is natural to question
the relative lack of automated identification aids in study of ecological
systems. 

Perhaps, an important concern 
%for the relatively low amount of research on automatic identification in the ecological domains, could be simply that, seemingly, such areas may not have been thought of as human-centric and hence the research was not need-driven. However, such a perception is now changing, and the importance of biodiversity studies is clearly being realized \cite{}. A more practical reason is the difficulty in data acquisition as compared to the more conventional domains. Importantly, an aspect 
which makes such research problems technically challenging, is the very large
amount of variations in the data. For instance, there are variations in
different entities of the same species, and also spatial and temporal variations
of the observed entities. This makes it difficult to model or represent each of
the relevant classes in a specific problem, or to develop good classification
approaches which can handle possible overlapping representations between
classes.

Having described the challenges for pattern recognition research in the
ecological domains, it must be acknowledged that the situation is not so grim.
The evolution of machine learning, vision and speech-acoustics research has
yielded some powerful mathematical tools such as graphical models, sparse
learning, and various advanced feature representations and classification
frameworks. 
As hinted above, these have been applied successfully in various areas such as
person identification, scene understanding, speech processing and recognition
etc.~, which also involve some of the above discussed concerns about variability
and resulting ambiguities. 

Indeed, the applications and adaptations of such frameworks have recently been
reported for automatic identification tasks in some domains of ecological
interest. Some example domains that employ audio-visual data include
identification and classification of birds, butterflies, plants etc.~, analysis
of animal and insect behaviour
\cite{leaf,butterfly,hierarchy,poof,brandes_automatedSoundRecording2008}. 

However, in general, such research is still in its nascent stages, and has much
scope for improvements in accuracy and the variety of scenarios. At the same time it
also provides much scope of intellectually challenging technical progress in
terms of developing new tools and frameworks, or adapting existing ones, so as
to address the new challenges. Another important perspective is that such
research can lead to possibilities of some system development, which we will
elaborate on subsequently. Some basic forms of such systems (e.g. website or
applications on mobile devices for data collection and analysis) have been setup
in recent years (e.g. \cite{birdsnap}). However, from a regional point of view,
such systems largely focus on the data from the western world, and there is
enough scope to develop similar or better systems for the Indian scenario. 

Having provided a general perspective, we acknowledge that each ecological
domain and the problems therein requires quite different approaches. In this
proposal, we focus on the problem of the detecting and classification of birds.
Our interest in considering this problem stems from the following reasons: 

\begin{itemize}

\item Our primary motivation to consider this problem is the challenge that it
poses in terms of the variety and necessarily unstructured nature of the data.
While, as indicated above, interest in this area has been growing, the
performance on existing approaches is very modest due to such practical
challenges. 

\item The varied nature of real data, in turn, brings about various technical
challenges such as exploring ways to extract/learn the discriminative features,
considering and modeling the local inter-relationships between visual bird-body
parts, developing frameworks which yields high similarity between similar
species, and making the failure cases to be graceful (i.e. more among similar
species and less as the specific difference increases), etc. 

\item We are also motivated to consider this problem from the perspective of
using both visual and acoustic modalities, which to our knowledge, has not been
reported. We believe that sight and sound, in the task of bird detection and
identification will compliment each other well and can be an important towards
improving the performance. 

\item Clearly, central to the problem is the availability of data in terms of
images/videos and sounds. Some standard public datasets for this purpose are
available for the purpose of benchmarking (e.g. \cite{cub}). However, these are
rather limited and region-specific to the western world (e.g. North America).
Also, there is further need for more realism in terms of scale, pose,
illumination variations in terms of visual data and background noise in terms of
acoustic data. Thus, over the course of this this project, we also plan to
collect data for Indian birds, both in terms of sight and sounds, and make it
publicly available.

\end{itemize}
 
\subsection{Objectives}
Based on the above discussions, we formulate below, some specific objectives. We
provide more details on these in section 3.  
\begin{enumerate}
\item Research and algorithm development for detection and identification of
birds from images and videos.
\item Research and algorithm development for detection and identification of
birds from acoustic data.
\item Integration of 1.~and 2.~above, into a common approach which can process audio-visual data. 
\item Constructing datasets of local bird species and making these available to the research community.
\item Eventually, establishing a web-based system which can use the automated
approaches, with some manual intervention, for further data collection and
management, resulting in continuous improvements.
\end{enumerate}

\section{Related work}
As mentioned above, the last few years has seen some modest but conscious effort
towards addressing the problem of automatic classification of birds. We briefly
discuss below, some related work, in the visual and acoustic domains.

\subsection{Fine-grained visual classification for bird identification}
From a visual point of view, the automated identification of bird species is one
of the important problems considered in an upcoming sub-area of fine-grained
visual classification (FGVC) in computer vision. Here, the term `fine-grained'
signifies that such classification problems are different (and, arguably,
harder) than the traditional visual classification problems. The latter consists
of classification at a coarser level of different object entities (e.g. type of
vehicle, objects, animals etc.), whereas FGVC is concerned about classification
within different species of a particular entity (e.g. car models, bird species
etc.).

One of the fundamental aspects in automated classification problems is the
\emph{representation} of the entities in different classes, in some objective
terms (or features) such as shape, texture etc. Clearly, in case of FGVC, global
representations of an entity such as the overall shape, morphology, geometry,
texture etc. will not be useful for the classification task, since entities
belonging to different classes would yield similar global representations. 

Hence, a more logical philosophy followed in existing approaches for FGVC is
that of considering the representation at a local level. These are typically
termed as part-based approaches \cite{dpm,hierarchy,poof,random,codebookfree},
wherein representative features are extracted from local parts of birds, and
these features are then used for further modeling or classification.  

For instance, in the work of \cite{random}, a variety of rectangular patches of
various widths and heights are sampled, and some features (e.g. SIFT, HOG
\cite{random,poof}) are extracted from these patches. Then, the approach
involves capturing the most discriminative patches, or pairs of patches, using a
random forest classifier. Another work \cite{codebookfree} also extracts
features from a variety of rectangular patches (templates), but then follows a
different direction of computing response maps by matching each training image
with the templates from all the testing images. Some local regions from these response
maps are then pooled to form histograms, which are then used for classification
via an support vector machine (SVM). The rationale behind this is that templates
from a bird species is more likely to match similar bird species than the other.
In these methods, there is no consideration of the structure or the relationship
of the parts of the object, and there is also a possibility of including some
background regions in the rectangular patches, while computing the features. 

In another work \cite{hierarchy}, a semi-automatic approach is followed to
consider the structure of the object (e.g. bird). First, the bird is segmented
from the background through manual intervention. This
foreground is then segmented into smaller regions, each one of which
approximately covers semantic parts of the bird (e.g. beak, nape etc.), based on
some off-line landmark information provided for each part. Now, feature
descriptors are computed from these segmented regions, and a codebook is learned
by pooling the descriptors, which is then used for classification using an SVM.
     
Realizing the importance of learning the most discriminative local features
(which to some extent was attempted in \cite{hierarchy} on unstructured
patches), the authors in \cite{poof}, propose to do the same at the level of
semantic features. Here, too the features (e.g. histogram of gradients etc.) are
extracted from patches, but the patches are structured around the points denoted
by landmarks. The discriminative regions are learned for every two classes,
based on the features in those regions using an SVM. Also, important in this work is the notion of
alignment of the regions across images (by scaling and rotating), based on the
landmark points. This is important so that the regions from which the features
are extracted coincide approximately in each image. 

The importance of image alignment for FGVC of birds is also stressed in
\cite{align}. In this work, the alignment is based on a global feature, viz. an
ellipse fitting on the bird shape. The alignment is then used to transfer the
local features and annotated landmarks of all images onto a common reference
frame.  

In another interesting work \cite{crow}, the primary goal is not the
classification, but to translate the objective and mathematically defined
features used in FGVC to semantic discriminative body parts, so as to form a
visual guidebook to know what discriminative parts to look for. The objective
features used in this work are those proposed in \cite{poof}.
 
\subsection{Bio-acoustic signal analysis for bird identification}

%\section*{Introduction to automatic birdsong classification}

Birdsong is one of nature's most well known sounds. Acoustic communication in
birds is one of the primary ways by which make their presence known to one
another. 
%Birds serve as important indicators of the health of ecosystems, and
%surveying bird populations can provide valueable inputs for conservation efforts.
%Recently, automatic monitoring of bird populations by means of pattern
%classification algorithms have received much attention. 
Most birds have
distinctive calls, which can be used to identify a particular species. Automatic
birdsong analysis provides several problems for the development of novel machine learning
and signal processing algorithms.

%\section*{A brief summary of existing techniques}

Sounds produced by birds can be complex and varied. A few broad categories of
sounds have been identified, which are common to most bird sounds
\cite{brandes_automatedSoundRecording2008}. But there are complexities in the
form of variations and combinations. There can be simple repetitions of the
sounds, or random sequences with no repetition.  Additionally, in field
recordings, there are other sounds, including calls from other birds, as well as
other natural and man-made sounds. All these factors make automatic birdsong
classification a non-trivial task. In addition to identifying different species,
automated systems have shown some success in identifying individuals
\emph{within} species \cite{kirschel_territorial2011}.

As in any pattern classification task, there are two stages in building an
automated birdsong classifier: feature extraction, followed by classification.
Features used in birdsong classification have ranged from simple features which
directly measure properties of the audio signal, to features used in automatic
human speech recognition and their variants \cite{somervuo_parametric2006,
tyagi_spectralEnsemble2006, graciarena_acousticFrontEnd2010, tan_sparse2012}. 
Similarly, the classification stage has used
techniques like Euclidean distance, Bayesian classifiers
\cite{lopes_largeSpecies2011}, hidden Markov models
\cite{chu_hmmBird2011, graciarena_acousticSequence2011},
information-theoretic measures \cite{wang_entropyBird2013}, Gaussian mixture models
\cite{lee_gmmImageShape2013}, neural networks\cite{mporas_automatedAcoustic2012}, 
decision trees \cite{neal_timeFreq2011}, and support vector machines
\cite{tan_sparse2012}.

% Should we add this here? This is to check git.
% President Barack Obama visited India, and had a good time at Republic day. The first lady didnt seem to enjoy as much.
% President Obama visited India
% Obamas car is the Beast.

%\section{Proposed contributions and objectives}
%Off-line captured images and signals for training
%Videos with sound for testing: Videos with natural visual and acoustic clutter 
%Tagging visual instance of birds and their identification
%Tagging audio instances of birds and their identification
%Correlating both
%Detection and identification


\section{Proposed methodology and project plan}

Based on the discussion of the related work, we notice some aspects which are
clearly important with respect to identification of birds.

\subsection{Visual classification}
\label{subsec:visual}

From a point of view of visual identification, such important aspects can be summarized as: a)
importance of considering local parts, b) formulating some objective feature
definitions on these parts, c) learning discriminative local features, d)
considering the structure and the relationship between the local parts, e)
importance of alignment, f) considering the availability of landmarks or manual
annotations. 

Some of the these aspects (e.g. a, b) are considered  in all of the existing
works, as discussed above, while other aspects (e.g. c-f)  are considered in
some but not all. In any case, given the adolescence of the area, there is still
much scope of further explorations on these aspects. Here, we propose and
briefly discuss some directions which we would like to explore: 

\begin{itemize}
\item Modeling relationships between all the parts via graphical models: While
individual parts or pair-wise relationships between parts are considered in
existing methods, higher level relationships between parts are not.
Thus, a possible direction for further exploration is modeling a higher level
body structure using graphical models (e.g. Markov random fields or conditional
random fields) \cite{mrfbook,hmrf,mrf,crf,crf1}, which have been employed in 
other contexts in computer vision (e.g. \cite{mrfpose}). 

\item Considering cases with or without landmarks/annotations: In a real-world
scenario, often we may not have annotations (or manual part marking). Thus, it
would be useful to develop methods for the task of bird detection, segmentation,
automatic part identification or annotation. Interestingly, this problem has not
been considered in any of the existing works. These may be seen as a
unsupervised low-level feature detection or segmentation problem, using a set of
landmarked or annotated examples.

\item Comparative analysis between low-level feature descriptors: There is no
standardization or review yet for FGVC of birds. In would be interesting and
informative, as a part of this project to work on such an exhaustive survey,
especially also because the area is gaining popularity in the vision and speech
community.

\item Dynamic kernels for fine-grained pattern analysis: For the efficient FGVC
of birds, an image 
is represented as set of local feature vectors. Since the numbers of local
feature vectors in the set is different for different images, this leads to
varying length pattern classification. This gives an opportunity to explore the
dynamic kernels~\cite{DK_2003_MK_Wallraven}, that are constructed between the a
pair of sets of local feature vectors, for FGVC. Some of the popular dynamic
kernels used in the varying length pattern classification using SVM are
intermediate matching kernel (IMK)~\cite{IMK_2014_DAD_CC}, pyramid match kernel
(PMK)~\cite{PMK_2007_Grauman}, spacial pyramid match
kernel~\cite{SPMK_2006_Lazebnik} etc. We can also explore these dynamic kernels
as similarity measures for matching bird images. This also gives a scope to
design dynamic kernels specific to FGVC task. 
\end{itemize}

\subsection{Birdsong identification}

\begin{itemize}
\item With respect to human speech, automatic identification of speakers have
demonstrated state-of-the-art performance using the so-called \emph{subspace
methods.} Here, the underlying idea is that most of the relevant information
needed for classification 
lies in a relatively low-dimensional space. The basis vectors of this space can
be learnt form training data \cite{dehak_ivector}. Similar subspace techniques
can be investigated for classification of bird sounds.
\item Again with respect to human speech, an important auditory cue for the
identification of speakers is the use of formant information. Formants are
resonances of the vocal tract. Bird vocalizations also show characteristic
formant information. Features that capture formant information (eg.~group delay
based representations \cite{padmanAllPoleGDelay, hegdeModgdf}) show great potential in
automatic identification of bird calls.

\item Dynamic kernels for analysis of birdsong: Similar to images, acoustic
signals are also represented as sets/sequences of local feature vectors. We
explore dynamic kernels mentioned in the section \ref{subsec:visual} for
birdsong identification as well as for matching and retrieval.  

\end{itemize}





\subsection{Feature selection and combination}

Many kinds of feature descriptors can be used in both audio and visual data, and
the effectiveness of feature descriptor varies with the problem domain. Thus, an
obvious direction of research is on fusing the information from the audio and visual domains.
These can include information-theoretic methods and multiple-kernel learning
\cite{LargeScaleMKL,mkl2}.

%direction is that of selecting the best image features and sound
%features for the pool of features using the different techniques including
%multiple kernel learning and combine the selected features from image domain and
%sound domain using information theoretic methods and multiple kernel learning. 

%\item Classifiers: There are various classifiers that can be explored for this
%task (e.g. \cite{duda}). Some well-known examples are Gaussian mixture models,
%neural networks, $K$-nearest neighbour classifier, support vector machines using
%Gaussian kernel, kernels for images and kernels for sound/speech signals,
%sparse-representation based classifiers \cite{sparse,sparse1}. Some of these
%classifiers can also be used to learn better discriminative features. \ \\ \ \\
%\item Building classifieers to identify birds: Information of the birds are available through the images and the sounds. We propose to identify the birds using following approaches:

Thus, based on the above discussion about various directions, we summarize the
methodological aspects of our proposal: 

\begin{itemize}
	\item Feature selection and region-of-interest
	detection/segmentation/annotation (the latter in case of non-annotated data)
	for image and acoustic data. 
    \item Classification of birds from images using different features and classifiers.  
    \item Classify the birds from their sounds using different features and classifiers.    
    \item Multimodal classification: use of both visual and acoustic data.
    \item Review of features and classifiers.
\end{itemize}


%selecting the most discriminative features

%Algorithm development for identification from off-line images and signals
%Algorithm development for detecting visual and audio instances in real videos
%Algorithms for identification using frames from real but simplistic videos

\section{Discussion: Future work}

Once a basic framework for multimodal classification of birds is ready, various
applications can be deployed. These can include, for instance, the following:

\begin{enumerate}
\item \textbf{Web-based system:} A website can be built which can provide an
easy-to-use interface to identify birds. Pictures or sounds can be uploaded to
the system, and the system identifies the species. The system will be such that
data labels can be collected as well. For example, the user can be provided with
an option to provide a class label (ie.~species name) and a confidence measure
when data is being uploaded. In this manner, a subset of the uploaded data can be
used as additional training data.

\item \textbf{Electronic field guide:} The huge popularity of handheld
devices naturally justify the creation of an app (Andriod or iOs) which can be
used for species identification. The algorithms can be scaled-down to run on 
smaller devices or can use a client-server model. Already some such apps exist
for North American birds.

\item \textbf{Biodiversity monitoring:} With large areas of forest having been cleared
for development projects, monitoring of biodiversity is crucial. An automated
monitoring system, consisting of audio-visual sensors, can be deployed at
locations of interest. Such a system can record different species of birds
automatically. Data collected in the long-term in this manner, can be a
valuable resource for biodiversity monitoring. Automatic classification can be
supplemented by manual intervention as needed.

\item \textbf{Further funding:} To extend upon this pilot work, proposals will be submitted for further
funding. This can be from government bodies, including Department of Science and
Technology, Department of Biotechnology, Department of Information Technology.

\end{enumerate}

%Further challenges:\\
%Handling audio-visual clutter\\
%Handling inter-class clutter for audio data\\
%Handling finer specific variations \ \\ \ \\
%Research in improvements for algorithms:\\
%Considering higher complexity by adding more classes \ \\ \ \\
%System development:\\
%Implementing a real system for bird identification\\
%A website where for maintaining a bird database and an online identification system\\
%Considering for generalizing for other wildlife\ \\

\section{Distribution of funds for this project}

An estimate of the funding for the project is given in Table \ref{tab:funding}.

\begin{table}[th]
\centering
\caption{Table of projected expenses. All amounts in lakhs INR.}
\begin{tabular}{|l|c|c|c|c|}
\hline
Items & Amount (Year 1) & Amount (Year 2) & Amount (Year 3) & Total\\
\hline
Manpower (2) & 4.0 & 4.0 & 4.0 & 12.0 \\
Imaging and audio equipment\tablefootnote{The DSLR camera/lenses included in the equipment will be contributed to the photography club.} & 5.0 & 3.0 & 0.0 & 8.0 \\
High-end computers\tablefootnote{Will be contributed to the institute HPC facility.} (2) & 3.0 & 3.0 & 0 & 6.0\\
Desktop computers (6) & 5.0 & 0.0 & 0 & 5.0  \\
Contingency & 0.5 & 0.5 & 1.0 & 2.0 \\
Travel & 0.5 & 0.5 & 1.0 & 2.0\\
\hline
Overall & 18.0 & 11.0 & 6.0 & \textbf{35.0} \\
\hline
\end{tabular}
\label{tab:funding}
\end{table}

\begin{thebibliography}{100}

%%% Arnavs bib begins here

\bibitem{monitor1}
G. Witmer, 
Wildlife population monitoring: Some practical considerations,
\emph{Wildlife Research}, 32, pp. 259--263, 2005. 

\bibitem{monitor2}
M. Bose, 
Common bird monitoring of India: A novel citizen science program, 
\emph{Science Reporter}, pp. 28--31, 2012. 

\bibitem{monitor3}
M. Ancrenaz, A. J. Hearn, J. Ross, R. Sollmann, A. Wilting, 
Handbook for wildlife monitoring using camera-traps,
\emph{BBEC Publication}, 2012. 

\bibitem{monitor4}
Annual report,
\emph{Cornell Lab of Ornithology}, 2011.

\bibitem{human1} 
M. Chesser, 
An investigation of human-error rates in wildlife photographic identification; 
Implications for the use of citizen scientists, 
\emph{Masters Theses, University of Massachusetts Amherst}, 2012. 

\bibitem{human2} 
N. J. Silvy, R. R. Lopez, M. J. Peterson, 
Wildlife marking techniques, 
\emph{Techniques for Wildlife Investigations and Management. Bethesda: The
Wildlife Society},
pp. 339--363, 2005. 

\bibitem{poof} 
T. Berg, P. Belhumeur, 
POOF: Part-based one-vs-one features for fine-grained categorization, 
face verification, and attribute estimation,
\emph{IEEE International Conference on Computer Vision and Pattern Recognition},
pp. 955--962, 2013. 

\bibitem{crow} 
T. Berg, P. Belhumeur, 
How do you tell a blackbird from a crow?, 
\emph{IEEE International Conference on Computer Vision},
pp. 729--736, 2014. 

\bibitem{dpm} 
N. Zhang, R. Farrell, F. Iandola, T. Darrell, 
Deformable part descriptors for fine-grained recognition and attribute prediction, 
\emph{IEEE International Conference on Computer Vision}, 2014. 

\bibitem{align} 
E. Gavves, B. Fernando, C. Snoek, A. Smeulders, T. Tuytelaars, 
Fine-grained categorization by alignments, 
\emph{IEEE International Conference on Computer Vision}, 2013. 

\bibitem{hierarchy} 
L. Xie, Q. Tian, R. Hong, S. Yan, B. Zhang, 
Hierarchical part matching for fine-grained visual categorization, 
\emph{IEEE International Conference on Computer Vision}, 2013. 

\bibitem{codebookfree} 
B. Yao, G. Bradski, L. Fei-Fei, 
A codebook-free and annotation-free approach for fine-grained image categorization, 
\emph{IEEE International Conference on Computer Vision and Pattern Recognition},
2012. 

\bibitem{random} 
B. Yao, A. Khosla, L. Fei-Fei, 
Combining randomization and discrimination for fine-grained image categorization,
\emph{IEEE International Conference on Computer Vision and Pattern Recognition}, 2011. 

\bibitem{cub} 
P. Welinder, S. Branson, T. Mita, C. Wah, F. Schroff, S. Belongie, P. Perona 
Caltech-UCSD Birds 200, 
\emph{Technical Report, California Institute of Technology},
CNS-TR-2010-001, 2010. 

\bibitem{birdsnap} 
BIRDSNAP: An Electronic Field Guide to Birds, \url{www.birdsnap.com}, 
\emph{Columbia University and the University of Maryland},
2014. 

\bibitem{leaf} 
S. G. Wu, F. S. Bao, E. Y. Xu, Y. Wang, Y. Chang, Q. Xiang,
A leaf recognition algorithm for plant classification using probabilistic neural network,
\emph{International Symposium on Signal Processing and Information Technology}, 2007. 

\bibitem{butterfly} 
J. Wang, K. Markert, M. Everingham, 
Learning models for object recognition from natural language descriptions, 
\emph{British Machine Vision Conference}, 2009. 

\bibitem{mrfbook} 
S. Z. Li, Markov random field modeling in computer vision, 
\emph{Springer-Verlag}, 1995. 

\bibitem{mrf} 
P. Kohli, C. Rother, 
Higher-order models in Computer Vision, 
\emph{O. Lezoray, L. Grady. (Eds.),
Image Processing and Analysing Graphs: Theory and Practice, CRC Press},
2012. 

\bibitem{hmrf} 
C. Wang, 
Distributed and higher-order graphical models, 
\emph{PhD Thesis, Ecole Centrale Paris,} 2011.

\bibitem{crf} 
X. He, R. Zemel, M. Carreira-Perpinan, 
Multiscale conditional random fields for image labeling, 
\emph{IEEE International Conference on Computer Vision and Pattern Recognition},
2004. 

\bibitem{crf1} 
J. Verbeek and B. Triggs, 
Scene segmentation with conditional random fields learned from partially labeled images, 
\emph{Conference on Neural Information Processing Systems}, 2007. 

\bibitem{mrfpose} 
M. Bray, P. Kohli, P. Torr, 
PoseCut: Simultaneous segmentation and 3D pose estimation of humans using dynamic graph-cuts,
\emph{European Conference on Computer Vision}, 2006. 

\bibitem{sparse} 
J. Wright, A.Y. Yang, A. Ganesh, S. Sastry, Y. Ma, 
Robust face recognition via sparse representation, 
\emph{IEEE Transactions on Pattern Analysis and Machine Intelligence}, 2009. 

\bibitem{sparse1} 
R. Rigamonti, M. A. Brown, V. Lepetit, 
Are sparse representations really relevant for image classification?, 
\emph{IEEE International Conference on Computer Vision and Pattern Recognition},
2011. 

\bibitem{duda} 
R. Duda, P. Hart, D. Stark., Pattern Classification, 
\emph{John Wiley and Sons Inc.}, 2000. 




%% Dileep's bib begins here

\bibitem{DK_2003_MK_Wallraven}
C. Wallraven and B. Caputo and A. Graf,
Recognition with Local Features: The Kernel Recipe,
\emph{Proceedings of the Ninth IEEE International Conference on Computer Vision},
2003


\bibitem{IMK_2014_DAD_CC} 
A. D. Dileep and C. Chandra Sekhar, 
{GMM}-Based Intermediate Matching Kernel for Classification of Varying Length 
Patterns of Long Duration Speech Using Support Vector Machines, 
\emph{IEEE Transactions on Neural Networks and Learning Systems}, 25(8),
2014. 

\bibitem{PMK_2007_Grauman}
K. Grauman and T. Darrell,
The Pyramid Match Kernel: Efficient Learning with Sets of Features,
\emph{Journal of Machine Learning Research},
8, 2007

\bibitem{SPMK_2006_Lazebnik} 
S. Lazebnik and C. Schmid and J. Ponce,
Beyond Bags of Features: Spatial Pyramid Matching for Recognizing Natural Scene Categories, 
\emph{IEEE International Conference on Computer Vision and Pattern Recognition}, 
2006


\bibitem{LargeScaleMKL}
S. Sonnenburg and G. R\"{a}tsch and C. Sch\"{a}fer and B. Sch\"{o}lkopf,
Large Scale Multiple Kernel Learning,
\emph{Journal of Machine Learning Research},
 7, 2006

\bibitem{mkl2}
A. Rakotomamonjy, F. R. Bach, S. Canu, and Y. Grandvalet, 
Simple{MKL},
\emph{Journal of Machine Learning Research}, 
9, 2008.













%%%%% Paddys bib begins here
\bibitem{brandes_automatedSoundRecording2008}
T. S. Brandes,  Automated sound recording and analysis techniques for
bird surveys and conservation, \emph{Bird Conservation International}, S1:18, 2008

\bibitem{kirschel_territorial2011}
A. Kirschel and M. Cody and Z. Harlow, and V. J. Promponas,
and E. Vallejo and C. E. Taylor,
Territorial dynamics of Mexican ant-thrushes Formicarius moniliger revealed by
individual recognition of their songs, \emph{Ibis}, 2:153, 2010

\bibitem{dehak_ivector}
N. Dehak, and P. Kenny and R. Dehak and P. Dumouchel and P. Ouellet,
\emph{IEEE Trans. Audio, Speech Lang. Process.}, 19:4, 2011

\bibitem{padmanAllPoleGDelay}
P. Rajan, and T. Kinnunen and C. Hanilci and J. Pohjalainen and P. Alku, 
Using group delay functions from all-pole models for speaker recognition,
\emph{Proc. Interspeech}, 2013

\bibitem{hegdeModgdf}
Significance of the Modified Group Delay Feature in Speech Recognition,
R. M. Hegde, and H. A. Murthy and V. R. Gadde,
\emph{IEEE Trans. Audio, Speech, Lang. Process.},
15:1, 2007

\bibitem{somervuo_parametric2006}
P. Somervuo and A. Harma and S. Fagerlund,
Parametric representations of bird sounds for automatic species recognition,
\emph{IEEE Trans. Audio, Speech Lang. Process.}, 14:6, 2006


\bibitem{tyagi_spectralEnsemble2006}
H. Tyagi, and R. M. Hegde and H. A. Murthy and A. Prabhakar,
Automatic identification of bird calls using spectral ensemble average voice prints
\emph{Proc. EUSIPCO}, 2006

\bibitem{graciarena_acousticFrontEnd2010}
M. Graciarena, and M. Delplanche, and E. Shriberg, and A. Stolcke, and L. Ferrer, 
Acoustic front-end optimization for bird species recognition,
\emph{Proc. ICASSP}, 2010

\bibitem{tan_sparse2012}
L. N. Tan, L. Ngee and K. Kaewtip and M. Cody and C. Taylor, and A. Alwan,
Evaluation of a Sparse Representation-Based Classifier For Bird Phrase
Classification Under Limited Data Conditions,
\emph{Proc. Interspeech}, 2012

\bibitem{lopes_largeSpecies2011}
M. Lopes and L. Gioppo and T. Higushi and C. Kaestner, and C. Silla 
and A. Koerich, 
Automatic bird species identification for large number of species,
\emph{IEEE International Symposium Multimedia}, 2011

\bibitem{chu_hmmBird2011}
W. Chu, and D. Blumstein
Noise robust bird song detection using syllable pattern-based hidden {M}arkov
models,
\emph{Proc. ICASSP}, 2011

\bibitem{graciarena_acousticSequence2011}
M. Graciarena, and M. Delplanche, and E. Shriberg, and A. Stolcke,
Bird species recognition combining acoustic and sequence modeling,
\emph{Proc. ICASSP}, 2011

\bibitem{wang_entropyBird2013}
N. Wang, and R. Hudson and L. Tan and C. Taylor and A. Alwan and K. Yao
Bird phrase segmentation by entropy-driven change point detection,
\emph{Proc. ICASSP}, 2013

\bibitem{lee_gmmImageShape2013}
C. Lee and S. Hsu and J. Shih and C. Chou, 
Continuous birdsong recognition using Gaussian mixture modeling of image shape features
\emph{IEEE Trans. Multimedia}, 2013

\bibitem{mporas_automatedAcoustic2012}
I. Mporas and T. Ganchev and O. Kocsis and N. Fakotakis and
O. Jahn and K. Riede and K. Schuchmann, 
Automated Acoustic Classification of Bird Species from Real-Field Recordings,
\emph{Proc. IEEE Conf.~Tools with Artificial Intelligence}, 2012

\bibitem{neal_timeFreq2011}
L. Neal, and F. Briggs, and R. Raich and X. Fern,
Time-frequency segmentation of bird song in noisy acoustic environments,
\emph{Proc. ICASSP}, 2011



















\end{thebibliography}
    
\end{document}
