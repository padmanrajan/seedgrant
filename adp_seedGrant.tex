\documentclass{article}
%\bibliographystyle{splncs}
\usepackage[dvips]{graphicx}
% comment
% comment 2
\evensidemargin 0.1cm 
\oddsidemargin 0.1cm 
\topmargin 0in 
\textwidth 16.5cm 
\headheight -0.5in 
\headsep 0in 
\textheight 10in 
\footskip 0.3in
%\footheight 0.1in 

\renewcommand{\baselinestretch}{1.5} 

\begin{document}

\title{Multimodal Identification of Birds from Visual and Acoustic Data}
\author{Arnav Bhavsar, Dileep A.~D.~, Padmanabhan Rajan\footnote{Author names
appear in alphabetic order.}\  \\Multimedia Analytics and Systems Group \\ 
School of Computing and Electrical Engineering\\
Indian Institute of Technology Mandi, India\\
\texttt{\{arnav, dileepad, padman\}@iitmandi.ac.in}}

\date{}
\maketitle
%\begin{abstract}
%\end{abstract}

\section{Introduction}

% this is a seed grant proposal

% this is my second comment to create more confusion.
% GIT is full of magic, and there is water dripping from our roof.

% This is added after Arnavs comment

% Hopefully the final test, this is.


%Biodiversity is central to existence of most natural ecosystems \cite{}, and clearly the impact of human settlements (notwithstanding the nobility of their cause) can create disturbances and large variations in the biodiversity. 
The realization about the importance of biodiversity preservation has provided impeteus to the scientific study of ecological systems, including those for monitoring the diversity, populations, distributions etc. \cite{monitor1,monitor2,monitor3,monitor4}. %Some of the primary factors in studying ecosystems is observation and identification \cite{}. 
The development of such scientific studies occurred well before the common availability of advanced technological aids. Thus, traditionally, the observation and identification process has been manual requiring much time and effort \cite{monitor4,human1,human2}. 

With the advent of the digital acquisition devices for capturing sights and sounds, the observation aspect has become easier, faster and more reliable. However, the identification component is still largely manual, with experts such as zoologists, ornithologists, botanists etc. manually labeling the acquired data. 

On the other hand, automated identification, labeling, annotating etc. has become commonplace in various human technological endeavors such as biometrics, surveillance, entertainment, navigation, sports etc. Clearly, given the nature of technical progress in more human-centric domains, it is natural to question the relative lack of automated identification aids in study of ecological systems. 

Perhaps, an important concern %for the relatively low amount of research on automatic identification in the ecological domains, could be simply that, seemingly, such areas may not have been thought of as human-centric and hence the research was not need-driven. However, such a perception is now changing, and the importance of biodiversity studies is clearly being realized \cite{}. A more practical reason is the difficulty in data acquisition as compared to the more conventional domains. Importantly, an aspect 
which makes such research problems technically challenging, is the very large amount of variations in the data. For instance, there are variations in different entities of the same species, and also spatial and temporal variations of the observed entities. This makes it difficult to model or represent each of the relevant classes in a specific problem, or to develop good classification approaches which can handle possible overlapping representations between classes.

Having described the challenges for pattern recognition research in the ecological domains, it must be acknowledged that the situation is not so grim. The evolution of machine learning, vision and speech-acoustics research has yielded some powerful mathematical tools have such as graphical models, sparse learning, and various advanced feature representation and classification frameworks. 
As hinted above, these have been applied successfully in various areas such as person identification, scene understanding, speech processing and recognition etc., which also involve some of the above discussed concerns about variability and resulting ambiguities. 

Indeed, the applications and adaptations of such frameworks have recently been reported for automatic identification tasks in some domains of ecological interests. Some example domains that employ audio-visual data include identification and classification of birds, butterflies, plants etc., analysis of animal and insect behaviour \cite{leaf,butterfly,hierarchy,poof,brandes_automatedSoundRecording2008}. 

However, in general, such research is still in its nascent stages, and has much scope for improvements in accuracy and variety of scenarios. At the same time it also provides much scope of intellectually challenging technical progress in terms of developing new tools and frameworks, or adapting existing ones, so as to address the new challenges. Another important perspective is that such research can lead to possibilities of some system development, which we will elaborate on subsequently. Some basic forms of such systems (e.g. website or applications on mobile devices for data collection and analysis) have been setup in recent years (e.g. \cite{birdsnap}). However, from a regional point of view, such systems largely focus on the data from the western world, and there is enough scope to develop similar or better systems for the Indian scenario. 

Having provided a general perspective, we acknowledge that each ecological domain and the problems therein requires quite different approaches. In this proposal, we focus on the problem of the detecting and classification of birds. Our interest in considering this problem stems from the following reasons: 
\begin{itemize}
\item Our primary motivation to consider this problem is the challenge that it poses in terms of the variety and necessarily unstructured nature of the data. While, as indicated above, interest in this area has been growing, the performance on existing approaches is very modest due to such practical challenges. 
\item The varied nature of real data, in turn, brings about various technical challenges such as exploring ways to extract/learn the discriminative features, considering and modeling the local inter-relationships between visual bird-body parts, developing frameworks which yields high similarity between similar species, and making the failure cases to be graceful (i.e. more among similar species and less as the specific difference increases), xxxx, etc. 
\item We are also motivated to consider this problem from the perspective of using both visual and acoustic modalities, which to our knowledge, has not been reported. We believe that sight and sound, in the task of bird detection and identification will compliment each other well and can be an important towards improving the performance. 
\item Clearly, central to the problem is the availability of data in terms of images/videos and sounds. Some standard public datasets for this purpose are available for the purpose of benchmarking (e.g. \cite{cub}). However, these are rather limited and region-specific to the western world (e.g. North America). Also, there is further need for more realism in terms of scale, pose, illumination variations in terms of visual data and xxxx in terms of acoustic data. Thus, over the course of this this project, we also plan to collect data for Indian birds, both in terms of sight and sounds, and make it publicly available.
\end{itemize}
 
\subsection{Objectives}
Based on the above discussions, we formulate below, some specific objectives. We provide more details on these in section 3.  
\begin{enumerate}
\item Research and algorithm development for detection and identification of birds from images and videos
\item Research and algorithm development for detection and identification of birds from acoustic data
\item Integration of 1. and 2. above, into a common approach which can process audio-visual data. 
\item Constructing datasets of local bird species and making these available to the research community.
\item Eventually, establishing a web-based system which can use the automated
approaches, with some manual intervention, for further data collection and management, resulting in continuous improvements.
\end{enumerate}

\section{Related work}
As mentioned above, the last few years has seen some modest but conscious effort towards addressing the problem of automatic classification of birds. We briefly discuss below, some related work, in the visual and acoustic domains.

\subsection{Fine-grained visual classification for bird identification}
From a visual point of view, the automated identification of bird species is one of the important problems considered in an upcoming sub-area of fine-grained visual classification (FGVC) in computer vision. Here, the term `fine-grained' signifies that such classification problems are different (and, arguably, harder) than the traditional visual classification problems. The latter consists of classification at a coarser level of different object entities (e.g. type of vehicle, objects, animals etc.), whereas FGVC is concerned about classification within different species of a particular entity (e.g. car models, bird species etc.).

One of the fundamental aspects in automated classification problems is the \emph{representation} of the entities different classes, in some objective terms (or features) such as shape, texture etc. Clearly, in case of FGVC, global representations of an entity such as the overall shape, morphology, geometry, texture etc. will not be useful for the classification task, since entities belonging to different classes would yield similar global representations. 

Hence, a more logical philosophy followed in exisiting approaches for FGVC is that of considering the representation at a local level. These are typically termed as part-based approaches \cite{dpm,hierarchy,poof,random,codebookfree}, wherein representative features are extracted from local parts of birds, and these features are then used for further modelling or classification.  

For instance, in the work of \cite{random}, a variety of rectangular patches of various widths and heights are sampled, and some features (e.g. SIFT, HOG \cite{random,poof}) are extracted from these patches. Then, the approach involves capturing the most discriminative patches, or pairs of patches, using a random forest classifier. Another work \cite{codebookfree} also extracts features from a variety of rectangular patches (templates), but then follows a different direction, of computing response maps by matching each training image with the templates from all the images. Some local regions from these response maps are then pooled to form histograms, which are then used for classification via an support vector machine (SVM). The rationale behind this is that templates from a bird species is more likely to match similar bird species than the other. In these methods, there is no consideration of the structure or the relationship of the parts of the object, and there is also a possibility of including some background regions in the rectangular patches, while computing the features. 

In another work \cite{hierarchy}, a semi-automatic approach is followed to consider the structure of the object (e.g. bird). First, an method which uses a small manual interaction is used to segment the bird from the background. This foreground is then segmented into smaller regions, each one of which appximately covers semantic parts of the bird (e.g. beak, nape etc.), based on some offline landmark information provided for each part. Now, feature descriptors are computed from these segmented regions, and a codebook is learned by pooling the descriptors, which is then used for classification using an SVM.
     
Realizing the importance of learning the most discriminative local features (which to some extent was attempted in \cite{hierarchy} on unstructured patches), the authors in \cite{poof}, propose to do the same at the level of semantic features. Here, too the features (e.g. histogram of gradients etc.) are extracted from patches, but the patches are structured around the points denoted by landmarks. The discriminative regions are learned for every two classes, based on the features in those regions, using an SVM, where the SVM also assign weights to each of the region. Also, important in this work is the notion of alignment of the regions across images (by scaling and rotating), based on the landmark points. This is important so that the regions from which the features are extracted coincide approximately in each image. 

The importance of image alignment for FGVC of birds is also stressed in \cite{align}. In this work, the alignment is based on a global feature, viz. an ellipse fitting on the bird shape. The alignment is then used to transfer the local features and annotated landmarks of all images onto a common reference frame.  

In another interesting work \cite{crow}, the primary goal is not the classification, but to translate the objective and mathematically defined features used in FGVC to semantic discriminative body parts, so as to form a visual guidebook to know what discriminative parts to look for. The objective features used in this work are those proposed in \cite{poof}.
 
\subsection{Bio-acoustic signal analysis for bird identification}

%\section*{Introduction to automatic birdsong classification}

Birdsong is one of nature's most well known sounds. Acoustic communication in
birds is one of the primary ways by which make their presence known to one
another. 
%Birds serve as important indicators of the health of ecosystems, and
%surveying bird populations can provide valueable inputs for conservation efforts.
Recently, automatic monitoring of bird populations by means of pattern
classification algorithms have received much attention. Most birds have
distinctive calls, which can be used to identify a particular species. Automatic
birdsong analysis provides several problems for the development of novel machine learning
and signal processing algorithms.

%\section*{A brief summary of existing techniques}

Sounds produced by birds can be complex and varied. A few broad categories of
sounds have been identified, which are common to most bird sounds
\cite{brandes_automatedSoundRecording2008}. But there are complexities in the
form of variations and combinations. There can be simple repetitions of the
sounds, or random sequences with no repetition.  Additionally, in field
recordings, there are other sounds, including calls from other birds, as well as
other natural and man-made sounds. All these factors make automatic birdsong
classification a non-trivial task. In addition to identifing different species,
automatated systems have shown some success in identifying individuals
\emph{within} species \cite{kirschel_territorial2011}.

As in any pattern classification task, there are two stages in building an
automated birdsong classifier: feature extraction, followed by classification.
Features used in birdsong classification have ranged from simple features which
directly measure properties of the audio signal, to features used in automatic
human speech recognition and their variants \cite{somervuo_parametric2006,
tyagi_spectralEnsemble2006, graciarena_acousticFrontEnd2010, tan_sparse2012}. 
Similarly, the classification stage has used
techniques like Eucledian distance, Bayesian classifiers
\cite{lopes_largeSpecies2011}, hidden Markov models
\cite{chu_hmmBird2011, graciarena_acousticSequence2011},
information-theoretic measures \cite{wang_entropyBird2013}, Gaussian mixture models
\cite{lee_gmmImageShape2013}, neural networks\cite{mporas_automatedAcoustic2012}, 
decision trees \cite{neal_timeFreq2011}, and support vector machines
\cite{tan_sparse2012}.

% Should we add this here? This is to check git.
% President Barack Obama visited India, and had a good time at Republic day. The first lady didnt seem to enjoy as much.
% President Obama visited India
% Obamas car is the Beast.

%\section{Proposed contributions and objectives}
%Off-line captured images and signals for training
%Videos with sound for testing: Videos with natural visual and acoustic clutter 
%Tagging visual instance of birds and their identification
%Tagging audio instances of birds and their identification
%Correlating both
%Detection and identification


\section{Proposed methodology and project plan}

\subsection{Visual classification}

Based on the discussion of the related work, we notice some aspects which are clearly important with respect to identification of birds. From a visual identification point of view, these can be summarized as: a) Important of considering local parts, b) formulating some objective feature definitions on these parts, c) learning most discriminative local features, d) Considering the structure and the relationship between the local parts, e) Importance of alignment, f) considering the availability of landmarks or manual annotations. 

Some of the these aspects (e.g. a, b) are considered  in all of the exisiting works, as discussed above, while other aspects (e.g. c-f)  are considered in some but not all. In any case, given the adolesence of the area, there is still much scope of further explorations on these aspects. Here, we propose and briefly discuss some directions which we would like to explore: 
\begin{itemize}
\item Modelling relationships between all the parts via graphical models: While individual parts or pair-wise relationships between parts are considered in existing state-of-the-art, higher level relationships between parts are not. Thus, a possible direction for further exploration is modeling a higher level body structure using graphical models (Markov random fields or conditional random fields) \cite{mrfbook,hmrf,mrf,crf,crf1}. Indeed, such higher order models between parts do exists in computer vision but in a different context of human pose/activity recognition \cite{mrfpose}. We would like to explore this in our context of FGVC for birds.
\item Considering cases with or without landmarks/annotations: In a real-world scenario, often we may not have annotations (or manual part marking). Thus, it would be useful to develop methods for such cases. Thus, the task of bird segmentation or automatic part identification are also involved in such a scenario. Interestingly, this issue has not been considered in any of the existing works. These may be seen as a unsupervised low-level feature detection or segmentation problem, or learning the segmentation labels or feature characteristics from a high confidence landmarked or annotated examples.
\item With respect to human speech, automatic identification of speakers have
demonstrated state-of-the-art performance using the so-called \emph{subspace
methods.} Here, the underlying idea is that most of the relevant information
needed for classification 
lies in a relatively low-dimensional space. The basis vectors of this space can
be learnt form training data \cite{dehak_ivector}. Similar subspace techniques
can be investigated for classification of bird sounds.
\item Again with respect to human speech, an important auditory cue for the
identification of speakers is the use of formant information. Formants are
resonances of the vocal tract. Bird vocalizations also show characteristic
formant information. Features that capture formant information (eg. group delay
based representations \cite{padmanAllPoleGDelay, hegdeModgdf}) show great potential in
automatic identification of bird calls.
\item Feature selection and combination: Many kinds of feature descriptors can be used in both audio and visual data, and the effectiveness of feature descriptor varies with the problem domain. Thus, an interesting direction is that of selecting the best image features and sound features for the pool of features using the differnt techniques including multiple kernel learning and combine the selected features from image domain and sound domain using informatin theoretic methods and multiple kernel leraning. 
\item Comparative analysis between low-level feature descriptor: There is no standardization or review yet for FGVC of birds. In would be interesting and informative, as a part of this project to work on such an exhaustive survey, especially also because the area is gaining popularity in the vision and speech community.
\item Classifiers: There are various classifiers that can be explored for this task (e.g. \cite{duda}). Some well-known examples are Gaussian mixture models, neural networks, $K$-nearest neighbour classifieer, support vector machines using Gaussin kernel, kernels for images and kernels for sound/speech signals, sparse-representation based classifiers \cite{sparse,sparse1}. Some of these classifiers can also be used to learn better discriminative features. \ \\ \ \\
%\item Building classifieers to identify birds: Information of the birds are available through the images and the sounds. We propose to identify the birds using following approaches:
Thus, based on the above discussion about various directions, we summarize the methodlogical aspects of our proposal: 
\begin{itemize}
		\item Feature selection and feature detection/segmentation (the latter in case of non-annotated data) for image and sound data. 
    \item Classification of birds from images represented using the different features using different classifiers mentioned above.  
    \item Classify the birds from their sounds represented using the different features using different classifiers mentioned above. 
    \item Combine the classifier scores of the different classifiers built on both image and sound of birds.
    \item Multimodal classification: Use the both image and sound data to build classifier.
    \item Guaging how much role does features or classifiers play.
    \item Review of features and classifiers.
\end{itemize}
\end{itemize}


%selecting the most discriminative features

%Algorithm development for identification from off-line images and signals
%Algorithm development for detecting visual and audio instances in real videos
%Algorithms for identification using frames from real but simplistic videos

\section{Discussion: Future work}

Further challenges:\\
Handling audio-visual clutter\\
Handling inter-class clutter for audio data\\
Handling finer specific variations \ \\ \ \\
Research in improvements for algorithms:\\
Considering higher complexity by adding more classes \ \\ \ \\
System development:\\
Implementing a real system for bird identification\\
A website where for maintaining a bird database and an online identification system\\
%Considering for generalizing for other wildlife\ \\

\section{Distribution of funds for this project}

\begin{table}[th]
\centering
\caption{Table of projected expenses. All amounts in lakhs INR.}
\begin{tabular}{|l|r|}
\hline
Item & Amount \\
\hline
High-end computers (4) & 12.0 \\
Camera/imaging equipment & 2.0 \\
Recorders/audio equipment & 1.0 \\
Desktop computers (6) & 3.0 \\
Consumables/travel & 2.0 \\
\hline
Total & 20.0 \\
\hline
\end{tabular}
\end{table}

\begin{thebibliography}{100}

%%% Arnavs bib begins here

\bibitem{monitor1}
G. Witmer., \emph{Wildlife population monitoring: Some practical considerations}, Wildlife Research, 32, pp. 259--263, 2005. 

\bibitem{monitor2}
M. Bose., \emph{Common bird monitoring of India: A novel citizen science program}, Science Reporter, pp. 28--31, 2012. 

\bibitem{monitor3}
M. Ancrenaz, A.J. Hearn, J. Ross, R. Sollmann, A. Wilting., \emph{Handbook for wildlife monitoring using camera-traps}, BBEC Publication, 2012. 

\bibitem{monitor4}
Cornell Lab of Ornithology., \emph{Annual report 2011}, 2011. 

\bibitem{human1} 
M. Chesser., \emph{An investigation of human-error rates in wildlife photographic identification; Implications for the use of citizen scientists}, Masters Theses, University of Massachusetts Amherst, 2012. 

\bibitem{human2} 
N.J. Silvy, R.R. Lopez, M.J. Peterson., \emph{Wildlife marking techniques}, Techniques for Wildlife Investigations and Management. Bethesda: The Wildlife Society, pp. 339--363, 2005. 

\bibitem{poof} 
T. Berg, P. Belhumeur., \emph{POOF: Part-based one-vs-one features for fine-grained categorization, face verification, and attribute estimation}, IEEE International Conference on Computer Vision and Pattern Recognition, (CVPR 2013), pp. 955--962, 2013. 

\bibitem{crow} 
T. Berg, P. Belhumeur., \emph{How do you tell a blackbird from a crow?}, IEEE International Conference on Computer Vision, (ICCV 2014), pp. 729--736, 2014. 

\bibitem{dpm} 
N. Zhang, R. Farrell, F. Iandola, T. Darrell., \emph{Deformable part descriptors for fine-grained recognition and attribute prediction}, IEEE International Conference on Computer Vision, (ICCV 2013), 2014. 

\bibitem{align} 
E. Gavves, B. Fernando, C. Snoek, A. Smeulders, T. Tuytelaars., \emph{Fine-grained categorization by alignments}, IEEE International Conference on Computer Vision, (ICCV 2013), 2013. 

\bibitem{hierarchy} 
L. Xie, Q. Tian, R. Hong, S. Yan, B. Zhang., \emph{Hierarchical part matching for fine-grained visual categorization}, IEEE International Conference on Computer Vision, (ICCV 2013), 2013. 

\bibitem{codebookfree} 
B. Yao, G. Bradski, L. Fei-Fei., \emph{A codebook-free and annotation-free approach for fine-grained image categorization}, IEEE International Conference on Computer Vision and Pattern Recognition, (CVPR 2012), 2012. 

\bibitem{random} 
B. Yao, A. Khosla, L. Fei-Fei., \emph{Combining randomization and discrimination for fine-grained image categorization}, IEEE International Conference on Computer Vision and Pattern Recognition, (CVPR 2011), 2011. 

\bibitem{cub} 
P. Welinder, S. Branson, T. Mita, C. Wah, F. Schroff, S. Belongie, P. Perona. \emph{Caltech-UCSD Birds 200}, Technical Report, California Institute of Technology, CNS-TR-2010-001, 2010. 

\bibitem{birdsnap} 
Columbia University and the University of Maryland. \emph{BIRDSNAP: An Electronic Field Guide to Birds}, www.birdsnap.com, 2014. 

\bibitem{leaf} 
S.G. Wu, F.S. Bao, E.Y. Xu, Y. Wang, Y. Chang, Q. Xiang., \emph{A leaf recognition algorithm for plant classification using probabilistic neural network}, International Symposium on Signal Processing and Information Technology, 2007. 

\bibitem{butterfly} 
J. Wang, K. Markert, M. Everingham., \emph{Learning models for object recognition from natural language descriptions}, British Machine Vision Conference, (BMVC 2009), 2009. 

\bibitem{mrfbook} 
S.Z. Li., \emph{Markov random field modeling in computer vision}, Springer-Verlag, 1995. 

\bibitem{mrf} 
P. Kohli, C. Rother., \emph{Higher-order models in Computer Vision}, In: O. Lezoray, L. Grady. (Eds.), Image Processing and Analysing Graphs: Theory and Practice, CRC Press, 2012. 

\bibitem{hmrf} 
C. Wang., \emph{Distributed and higher-order graphical models}, PhD Thesis, Ecole Centrale Paris, 2011.

\bibitem{crf} 
X. He, R. Zemel, M.Carreira-Perpinan., \emph{Multiscale conditional random fields for image labeling}, IEEE International Conference on Computer Vision and Pattern Recognition, (CVPR 2004), pp. II-695--II-702, 2004. 

\bibitem{crf1} 
J. Verbeek and B. Triggs., \emph{Scene segmentation with conditional random fields learned from partially labeled images}, Conference on Neural Information Processing Systems (NIPS 2007), 2007. 

\bibitem{mrfpose} 
M. Bray, P. Kohli, P. Torr., \emph{PoseCut: Simultaneous segmentation and 3D pose estimation of humans using dynamic graph-cuts}, European Conference on Computer Vision, (ECCV 2006), 2006. 

\bibitem{sparse} 
J. Wright, A.Y. Yang, A. Ganesh, S. Sastry, Y. Ma., \emph{Robust face recognition via sparse representation}, IEEE Transactions on Pattern Analysis and Machine Intelligence, (PAMI 2009), 2009. 

\bibitem{sparse1} 
R. Rigamonti, M.A. Brown, V. Lepetit., \emph{Are sparse representations really relevant for image classification ?}, IEEE International Conference on Computer Vision and Pattern Recognition, (CVPR 2011), 2011. 

\bibitem{duda} 
R. Duda, P. Hart, D. Stark., \emph{Pattern Classification}, John Wiley and Sons Inc., 2000. 




%% Dileep's bib begins here







%%%%% Paddys bib begins here
\bibitem{brandes_automatedSoundRecording2008}
Scott Brandes, T, Automated sound recording and analysis techniques for
bird surveys and conservation, \emph{Bird Conservation International}, S1:18, 2008

\bibitem{kirschel_territorial2011}
Kirschel, Alexander NG and Cody, Martin L and Harlow, Zachary T and Promponas,
Vasilis J and Vallejo, Edgar E and Taylor, Charles E,
Territorial dynamics of Mexican Ant-thrushes Formicarius moniliger revealed by
individual recognition of their songs, \emph{Ibis}, 2:153, 2010

\bibitem{dehak_ivector}
Dehak, N. and Kenny, P. and Dehak, R. and Dumouchel, P. and Ouellet, P.,
\emph{IEEE Trans. Audio, Speech Lang. Process.}, 19:4, 2011

\bibitem{padmanAllPoleGDelay}
Rajan, Padmanabhan and Kinnunen, Tomi and Hanilci, Cemal and Pohjalainen, J and Alku, P
Using group delay functions from all-pole models for speaker recognition,
\emph{Proc. Interspeech}, 2013

\bibitem{hegdeModgdf},
Significance of the Modified Group Delay Feature in Speech Recognition,
Hegde, R. M. and Murthy, H. A. and Gadde, V. R.,
\emph{IEEE Trans. Audio, Speech, Lang. Process.},
15:1, 2007

\bibitem{somervuo_parametric2006}
Somervuo, Panu and Harma, Aki and Fagerlund, Seppo,
Parametric representations of bird sounds for automatic species recognition,
\emph{IEEE Trans. Audio, Speech Lang. Process.}, 14:6, 2006


\bibitem{tyagi_spectralEnsemble2006}
Tyagi, Hemant and Hegde, Rajesh M and Murthy, Hema A and Prabhakar, Anil
Automatic identification of bird calls using spectral ensemble average voice prints
\emph{Proc. EUSIPCO}, 2006

\bibitem{graciarena_acousticFrontEnd2010}
Graciarena, Martin and Delplanche, Michelle and Shriberg, Elizabeth and Stolcke, Andreas and Ferrer, Luciana
Acoustic front-end optimization for bird species recognition,
\emph{Proc. ICASSP}, 2010

\bibitem{tan_sparse2012}
Tan, Lee Ngee and Kaewtip, Kantapon and Cody, Martin L and Taylor, Charles E and
Alwan, Abeer,
Evaluation of a Sparse Representation-Based Classifier For Bird Phrase
Classification Under Limited Data Conditions,
\emph{Proc. Interspeech}, 2012

\bibitem{lopes_largeSpecies2011}
Lopes, Marcelo Teider and Gioppo, Lucas L and Higushi, Thiago T and Kaestner,
Celso AA and Silla, CN and Koerich, Alessandro L,
Automatic bird species identification for large number of species,
\emph{IEEE International Symposium Multimedia}, 2011

\bibitem{chu_hmmBird2011}
Chu, Wei and Blumstein, Daniel T
Noise robust bird song detection using syllable pattern-based hidden {M}arkov
models,
\emph{Proc. ICASSP}, 2011

\bibitem{graciarena_acousticSequence2011}
Graciarena, Martin and Delplanche, Michelle and Shriberg, Elizabeth and Stolcke,
Andreas,
Bird species recognition combining acoustic and sequence modeling,
\emph{Proc. ICASSP}, 2011

\bibitem{wang_entropyBird2013}
Wang, Ni-Chun and Hudson, Ralph E and Tan, Lee Ngee and Taylor, Charles E and Alwan, Abeer and Yao, Kung
Bird phrase segmentation by entropy-driven change point detection,
\emph{Proc. ICASSP}, 2013

\bibitem{lee_gmmImageShape2013}
Lee, Chang-Hsing and Hsu, Sheng-Bin and Shih, Jau-Ling and Chou, Chih-Hsun,
Continuous birdsong recognition using Gaussian mixture modeling of image shape features
\emph{IEEE Trans. Multimedia}, 2013

\bibitem{mporas_automatedAcoustic2012}
Mporas, Iosif and Ganchev, Todor and Kocsis, Otilia and Fakotakis, Nikos and
Jahn, Olaf and Riede, Klaus and Schuchmann, Karl-L,
Automated Acoustic Classification of Bird Species from Real-Field Recordings,
\emph{Proc. IEEE Conf.~Tools with Artificial Intelligence}, 2012

\bibitem{neal_timeFreq2011}
Neal, Lawrence and Briggs, Forrest and Raich, Raviv and Fern, Xiaoli Z,
Time-frequency segmentation of bird song in noisy acoustic environments,
\emph{Proc. ICASSP}, 2011






















\end{thebibliography}
    
\end{document}
